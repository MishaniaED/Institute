\documentclass[a4paper,14pt]{extarticle}
\usepackage[T2A]{fontenc}
\usepackage[utf8]{inputenc}
\usepackage[russian]{babel}

\usepackage[dvips]{graphicx}
\usepackage{color}
\usepackage[dvips]{hyperref}

\usepackage{setspace}
\usepackage{indentfirst}
\usepackage{textcomp}
\usepackage{ifthen}
\usepackage{calc}

\usepackage[cache=false]{minted}

\hypersetup{%
	unicode,%
	linkcolor=blue,
	colorlinks=true,%
%	pdfpagemode=FullScreen,%
%	pdfpagetransition=Dissolve,%
	pdftitle={Курсовая работа по дисциплине "Операционные системы и архитектура ПК"},%
}


\thispagestyle{empty}

\setlength{\voffset}{-.8in}
\setlength{\hoffset}{-.75in}
\addtolength{\textheight}{1.6in}
\addtolength{\textwidth}{1.5in}


\onehalfspacing

%Поменяйте фамилию, имя и отчество в команде FIO
\newcommand{\FIO}{Иванов Иван Иванович}

%Поменяйте название программного продукта в команде SOFTWARE
\newcommand{\SOFTWARE}{Вставить название}

%Укажите номер группы (210 или 211)
\newcommand{\GROUP}{210}

\setcounter{tocdepth}{2}
\begin{document}
\input{titulCW}
\newpage

\tableofcontents
\newpage
\
\part{Теоретическая}

\section{Название вопроса 1}
%Вставить название вопроса 1
%Описать теоретический вопрос № 1

\section{Название вопроса 2}
%Вставить название вопроса 2
%Описать теоретический вопрос № 2
\newpage

\part{Практическая}
%В каждом из подразделов нужно описать 
\section{Простые стандартные алгоритмы}
\subsection{Алгоритм Евклида} 
\subsubsection{Техническое задание}
Опишите постановку задачи

\subsubsection{Описание основного алгоритма}
Опишите алгоритм только не кодом. Реализация алгоритма представлена в подразделе \ref{code:evclid}.

\subsubsection{Руководство пользователя}
Опишите работу пользователя и вставьте скриншот

\subsection{Корректность расстановки скобок}
\subsubsection{Техническое задание}
Опишите постановку задачи

\subsubsection{Описание основного алгоритма}
Опишите алгоритм только не кодом

\subsubsection{Руководство пользователя}
Опишите работу пользователя и вставьте скриншот

\subsection{Переполнение}
\subsubsection{Техническое задание}
Опишите постановку задачи

\subsubsection{Описание основного алгоритма}
Опишите алгоритм только не кодом

\subsubsection{Руководство пользователя}
Опишите работу пользователя и вставьте скриншот

\subsection{Куб в виде суммы}
\subsubsection{Техническое задание}
Опишите постановку задачи

\subsubsection{Описание основного алгоритма}
Опишите алгоритм только не кодом

\subsubsection{Руководство пользователя}
Опишите работу пользователя и вставьте скриншот

\subsection{Числа Фибоначчи}
\subsubsection{Техническое задание}
Опишите постановку задачи

\subsubsection{Описание основного алгоритма}
Опишите алгоритм только не кодом

\subsubsection{Руководство пользователя}
Опишите работу пользователя и вставьте скриншот

\subsection{Сумма квадратов}
\subsubsection{Техническое задание}
Опишите постановку задачи

\subsubsection{Описание основного алгоритма}
Опишите алгоритм только не кодом

\subsubsection{Руководство пользователя}
Опишите работу пользователя и вставьте скриншот
\newpage

\section{Работа с файловой системой}
\subsection{Используемые системные объекты}
Опишите используемые в лабораторной системные объекты: что, зачем, как используется.

\subsection{Реализация ls}
\subsubsection{Техническое задание}
Опишите постановку задачи

\subsubsection{Описание основного алгоритма}
Опишите алгоритм только не кодом

\subsubsection{Руководство пользователя}
Опишите работу пользователя и вставьте скриншот

\subsection{Реализация ls без папок}
\subsubsection{Техническое задание}
Опишите постановку задачи

\subsubsection{Описание основного алгоритма}
Опишите алгоритм только не кодом

\subsubsection{Руководство пользователя}
Опишите работу пользователя и вставьте скриншот

\subsection{Жёсткая ссылка}
\subsubsection{Техническое задание}
Опишите постановку задачи

\subsubsection{Описание основного алгоритма}
Опишите алгоритм только не кодом

\subsubsection{Руководство пользователя}
Опишите работу пользователя и вставьте скриншот

\subsection{Символическая ссылка}
\subsubsection{Техническое задание}
Опишите постановку задачи

\subsubsection{Описание основного алгоритма}
Опишите алгоритм только не кодом

\subsubsection{Руководство пользователя}
Опишите работу пользователя и вставьте скриншот

\subsection{Сортировка файлов}
\subsubsection{Техническое задание}
Опишите постановку задачи

\subsubsection{Описание основного алгоритма}
Опишите алгоритм только не кодом

\subsubsection{Руководство пользователя}
Опишите работу пользователя и вставьте скриншот
\newpage

\section{Работа с процессами}
\subsection{Используемые системные объекты}
Опишите используемые в лабораторной системные объекты: что, зачем, как используется.

\subsection{Порождение процессов}
\subsubsection{Техническое задание}
Опишите постановку задачи

\subsubsection{Описание основного алгоритма}
Опишите алгоритм только не кодом

\subsubsection{Руководство пользователя}
Опишите работу пользователя и вставьте скриншот

\subsection{Иерархия процессов}
\subsubsection{Техническое задание}
Опишите постановку задачи

\subsubsection{Описание основного алгоритма}
Опишите алгоритм только не кодом

\subsubsection{Руководство пользователя}
Опишите работу пользователя и вставьте скриншот

\subsection{Подсчёт факториалов}
\subsubsection{Техническое задание}
Опишите постановку задачи

\subsubsection{Описание основного алгоритма}
Опишите алгоритм только не кодом

\subsubsection{Руководство пользователя}
Опишите работу пользователя и вставьте скриншот
\newpage

\section{Работа с семафорами}
\subsection{Используемые системные объекты}
Опишите используемые в лабораторной системные объекты: что, зачем, как используется.

\subsection{Задача читателей и писателей}
\subsubsection{Техническое задание}
Опишите постановку задачи

\subsubsection{Описание основного алгоритма}
Опишите алгоритм только не кодом

\subsubsection{Руководство пользователя}
Опишите работу пользователя и вставьте скриншот

\subsection{Семафоры и разделяемая память}
\subsubsection{Техническое задание}
Опишите постановку задачи

\subsubsection{Описание основного алгоритма}
Опишите алгоритм только не кодом

\subsubsection{Руководство пользователя}
Опишите работу пользователя и вставьте скриншот
\newpage

\section{Работа с каналами и сообщениями}
\subsection{Используемые системные объекты}
Опишите используемые в лабораторной системные объекты: что, зачем, как используется.

\subsection{Численное нахождение интеграла}
\subsubsection{Техническое задание}
Опишите постановку задачи

\subsubsection{Описание основного алгоритма}
Опишите алгоритм только не кодом

\subsubsection{Руководство пользователя}
Опишите работу пользователя и вставьте скриншот

\subsection{Нахождение корней кубического уравнения}
\subsubsection{Техническое задание}
Опишите постановку задачи

\subsubsection{Описание основного алгоритма}
Опишите алгоритм только не кодом

\subsubsection{Руководство пользователя}
Опишите работу пользователя и вставьте скриншот
\newpage

\section{Сетевое программирование}
\subsection{Используемые системные объекты}
Опишите используемые в лабораторной системные объекты: что, зачем, как используется.

\subsection{Подсчёт слов}
\subsubsection{Техническое задание}
Опишите постановку задачи

\subsubsection{Описание основного алгоритма}
Опишите алгоритм только не кодом

\subsubsection{Руководство пользователя}
Опишите работу пользователя и вставьте скриншот

\subsection{Чат}
\subsubsection{Техническое задание}
Опишите постановку задачи

\subsubsection{Описание основного алгоритма}
Опишите алгоритм только не кодом

\subsubsection{Руководство пользователя}
Опишите работу пользователя и вставьте скриншот
\newpage

\section{ПО <<\SOFTWARE>>}

\subsection{Техническое задание}
Требуется вставить техническое задание на Ваше ПО. 

Описать функциональные возможности приложения 

Описать входные данные, принимаемые приложением

\subsection{Структура приложения}
Описывается логика работы приложения, основые блоки и модули, а также взаимодействие между ними.
Перечислить разработанные модули и функции.  
Описать последовательность вызовов функций. 

\subsection{Основные алгоритмы}
Здесь приводятся алгоритмы, используемые автором

Описать работу каждой функции: входные и выходные параметры, схему работы (блок-схема, псевдокод или словесное описание).

Не в виде кода!!!

\subsection{Руководство пользователя}
Описать интерфейс приложения и взаимодействие с пользователем.

Привести пример корректной работы приложения (со скриншотами).

Описать реакцию приложения на аномалии входных данных.

\section{Список использованных источников}
Здесь должен быть список литературы, статей и прочих материалов, использованных в работе. 
\newpage

\section{Приложение}
Здесь приводится код. Каждый файл в своём подразделе.

\subsection{Алгоритм Евклида}\label{code:evclid}
\centerline{\textbf{Файл \texttt{main.c}}}
\begin{minted}{C}
#include<stdio.h>
int main(){	
	printf("hello, world!\n");
	return 0;
}
\end{minted}
\hrulefill

\subsection{Корректность расстановки скобок}\label{code:scobe}
\centerline{\textbf{Файл \texttt{main.c}}}
\begin{minted}{C}
#include<stdio.h>
int main(){	
	printf("hello, world!\n");
	return 0;
}
\end{minted}
\hrulefill

\subsection{Переполнение}\label{code:overfull}
\centerline{\textbf{Файл \texttt{main.c}}}
\begin{minted}{C}
#include<stdio.h>
int main(){	
	printf("hello, world!\n");
	return 0;
}
\end{minted}
\hrulefill

\subsection{Куб в виде суммы}\label{code:cube}
\centerline{\textbf{Файл \texttt{main.c}}}
\begin{minted}{C}
#include<stdio.h>
int main(){	
	printf("hello, world!\n");
	return 0;
}
\end{minted}
\hrulefill

\subsection{Числа Фибоначчи}\label{code:fib}
\centerline{\textbf{Файл \texttt{main.c}}}
\begin{minted}{C}
#include<stdio.h>
int main(){	
	printf("hello, world!\n");
	return 0;
}
\end{minted}
\hrulefill

\subsection{Сумма квадратов}\label{code:summ}
\centerline{\textbf{Файл \texttt{main.c}}}
\begin{minted}{C}
#include<stdio.h>
int main(){	
	printf("hello, world!\n");
	return 0;
}
\end{minted}
\hrulefill

\subsection{Реализация ls}\label{code:ls}
\centerline{\textbf{Файл \texttt{main.c}}}
\begin{minted}{C}
#include<stdio.h>
int main(){	
	printf("hello, world!\n");
	return 0;
}
\end{minted}
\hrulefill

\subsection{Реализация ls без папок}\label{code:ls-cat}
\centerline{\textbf{Файл \texttt{main.c}}}
\begin{minted}{C}
#include<stdio.h>
int main(){	
	printf("hello, world!\n");
	return 0;
}
\end{minted}
\hrulefill

\subsection{Жёсткая ссылка}\label{code:hardlink}
\centerline{\textbf{Файл \texttt{main.c}}}
\begin{minted}{C}
#include<stdio.h>
int main(){	
	printf("hello, world!\n");
	return 0;
}
\end{minted}
\hrulefill

\subsection{Символическая ссылка}\label{code:symlink}
\centerline{\textbf{Файл \texttt{main.c}}}
\begin{minted}{C}
#include<stdio.h>
int main(){	
	printf("hello, world!\n");
	return 0;
}
\end{minted}
\hrulefill

\subsection{Сортировка файлов}\label{code:sort}
\centerline{\textbf{Файл \texttt{main.c}}}
\begin{minted}{C}
#include<stdio.h>
int main(){	
	printf("hello, world!\n");
	return 0;
}
\end{minted}
\hrulefill

\subsection{Порождение процессов}\label{code:process}
\centerline{\textbf{Файл \texttt{main.c}}}
\begin{minted}{C}
#include<stdio.h>
int main(){	
	printf("hello, world!\n");
	return 0;
}
\end{minted}
\hrulefill

\subsection{Иерархия процессов}\label{code:hierarchy}
\centerline{\textbf{Файл \texttt{main.c}}}
\begin{minted}{C}
#include<stdio.h>
int main(){	
	printf("hello, world!\n");
	return 0;
}
\end{minted}
\hrulefill

\subsection{Подсчёт факториалов}\label{code:factorial}
\centerline{\textbf{Файл \texttt{main.c}}}
\begin{minted}{C}
#include<stdio.h>
int main(){	
	printf("hello, world!\n");
	return 0;
}
\end{minted}
\hrulefill

\subsection{Задача читателей и писателей}\label{code:read-write}
\centerline{\textbf{Файл \texttt{main.c}}}
\begin{minted}{C}
#include<stdio.h>
int main(){	
	printf("hello, world!\n");
	return 0;
}
\end{minted}
\hrulefill

\subsection{Семафоры и разделяемая память}\label{code:sem-shm}
\centerline{\textbf{Файл \texttt{main.c}}}
\begin{minted}{C}
#include<stdio.h>
int main(){	
	printf("hello, world!\n");
	return 0;
}
\end{minted}
\hrulefill

\subsection{Численное нахождение интеграла}\label{code:pi}
\centerline{\textbf{Файл \texttt{main.c}}}
\begin{minted}{C}
#include<stdio.h>
int main(){	
	printf("hello, world!\n");
	return 0;
}
\end{minted}
\hrulefill

\subsection{Нахождение корней кубического уравнения}\label{code:cube-equation}
\centerline{\textbf{Файл \texttt{main.c}}}
\begin{minted}{C}
#include<stdio.h>
int main(){	
	printf("hello, world!\n");
	return 0;
}
\end{minted}
\hrulefill

\subsection{Подсчёт слов}\label{code:wordcount}
\centerline{\textbf{Файл \texttt{main.c}}}
\begin{minted}{C}
#include<stdio.h>
int main(){	
	printf("hello, world!\n");
	return 0;
}
\end{minted}
\hrulefill

\subsection{Чат}\label{code:chat}
\centerline{\textbf{Файл \texttt{main.c}}}
\begin{minted}{C}
#include<stdio.h>
int main(){	
	printf("hello, world!\n");
	return 0;
}
\end{minted}
\hrulefill

\subsection{ПО <<\SOFTWARE>>}\label{code:soft}
\centerline{\textbf{Файл \texttt{main.c}}}
\begin{minted}{C}
#include<stdio.h>
int main(){	
	printf("hello, world!\n");
	return 0;
}
\end{minted}
\hrulefill

\end{document}
